\chapter{Conclusion and future work}
\label{conclusion}
\section{The big picture}
We started this thesis with three research questions and the objective to provide a methodology to distinguish between three modeling notations and combine them in a super-model. In Chapter \ref{chapter:indicators}, a set of characteristics was formed to describe and identify processes. Starting with BPMN, we achieved to define matching processes capturing \textit{routine} work with \textit{predictable} output, ultimately resulting in as much automation as possible. In contrast, CMMN is appropriate for the opposite type of work: \textit{agile} with \textit{emerging} tasks from circumstances that can be \textit{partly automated} as well as \textit{partly predefined}. The main difference between BPMN and CMMN is the focus on workflow execution (BPMN) and supporting knowledge workers in their daily, mostly manual, business (CMMN). DMN, on the other side, is a convenient tool to cope with data intensive processes. The reason why it is called \textit{Decision} Model and Notation can be derived from the data-centric aspect. Taken into account that a preceding process documentation is comprised of a spreadsheet, eventually this spreadsheet should result in a certain output, aiding in a decision-making process. DMN is a tool that aggregates data and results in a decision. It complements data- or decision-intensive processes. However, we saw in the Chapter \ref{chapter:case_study} that data-intensive processes cannot always be complemented with DMN or even completely modeled in this notation. The eKulturPortal is a data driven application. Its first milestone is comprised of features dealing with the creation and administration of database objects such as companies, venues and users. In this case, the indicators did not recommend DMN to handle these data processes, since the type of data did not match with DMN tables. An overview of all the indicators defined in Chapter \ref{chapter:indicators} can be found in Table \ref{tab:indicators}. \\
 
% Please add the following required packages to your document preamble:
% \usepackage{booktabs}
% \usepackage{graphicx}
% \usepackage[normalem]{ulem}
% \useunder{\uline}{\ul}{}
\begin{table}[h!]
\centering
\resizebox{\textwidth}{!}{%
\begin{tabular}{@{}lllll@{}}
\toprule
Categories                   & Weighting & BPMN Indicators                                 & CMMN Indicators                                                                                & DMN Indicators                                                                                      \\ \midrule
Documentation style          & 1         & Directives                                      & \begin{tabular}[c]{@{}l@{}}Descriptions of best practices\\ and recommendations\end{tabular}   & Spreadsheets                                                                                        \\
Preceding process map        & 1         & Flowchart                                       & Cluster                                                                                        & Decision trees                                                                                      \\
Characteristics of work      & 2         & routine, predictable, automatable               & \begin{tabular}[c]{@{}l@{}}Agile, emerging, \\ partly automatable\end{tabular}                 & \begin{tabular}[c]{@{}l@{}}decision-intensive, data handling,\\ decison-making\end{tabular}         \\
Characteristics of process   & 2         & rigid, predefined, workflow-centric             & \begin{tabular}[c]{@{}l@{}}adaptive, partly predefined, \\ knowledge-centric\end{tabular}      & \begin{tabular}[c]{@{}l@{}}data-centric, data processing,\\ complementary\end{tabular}              \\
Characteristics of decisions & 2         & Simple (either / or)                            & Stateful (transition form one to another state)                                                & Complex                                                                                             \\
Control flow                 & 1         & Definite control flow, required                 & Indefinite control flow, optional                                                              & Dependencies, required                                                                              \\
Intervention at run-time     & 1         & No                                              & Yes                                                                                            & Yes                                                                                                 \\
Objective                    & 2         & Automated workflow execution                    & Support manual work                                                                            & Automated data processing                                                                           \\
Type of process              & 2         & Business process                                & Case                                                                                           & Decisions                                                                                           \\
Typical application          & 1         & Billing process, Accounting, Assembly-line work & \begin{tabular}[c]{@{}l@{}}Reviews, Medical attendance,\\ Managing and organising\end{tabular} & \begin{tabular}[c]{@{}l@{}}Calculation of discount rates, salary,\\ choosing assignees\end{tabular} \\ \midrule
Points per Indicator        & \multicolumn{4}{c}{15}                                                                                                                                                                                                                                             \\ \bottomrule
\end{tabular}%
}
\caption{An overview of the indicators defined in Chapter \ref{chapter:indicators}.}
\label{tab:indicators}
\end{table}
After the definition of indicators, the combination aspect was examined in Chapter \ref{chapter:combination}. Each notation's ability to link different notations within the model was presented. We found out, that DMN has no designated interface to any of the two modeling notations, whereas BPMN and CMMN have interfaces to link with any of the three notations. \enlargethispage{1\baselineskip}\newpage BPMN features a \textit{Call Activity} and a \textit{Business Rule Task}, both referencing either different BPMN or CMMN processes or - in terms of a \textit{Business Rule Task} - referencing to a DMN table. CMMN features designated \textit{Business Process Tasks} and \textit{Case Tasks}, as well as \textit{Decision Tasks} that reference to the according processes, ultimately supporting the case worker with automated task execution. As we saw again in this chapter, DMN is a complementary notation without a practical use case for a standalone model. \\
With these two chapters in mind, Chapter \ref{chapter:case_study} showed practical applications of the indicators and how to combine models eventually. A key process, the \textit{Company audit}, was dissected in a bottom-up approach with a top-down view, as \cite{Verner2004} recommended and endorsed by different best practices. Starting with a process map, the process documentation has lead us through the discovery process creating sub-processes according to the \textit{Separation of Concerns} paradigm. Each process captures a logical unit, making it easier and clearer to model processes with a broad variety of deliberations. To orchestrate all the sub-processes, a super-process was created with the ability to call each logical unit upon request. As the \textit{Company Audit} process features \textit{agile} characteristics and the objective to \textit{support manual work}, the recommendation for this super-model was CMMN. The conclusion shows that the mentioned combinations allow modeling all references with their designated interfaces along with CMMN's functions to report states and to listen to user events, such as an abortion of the case execution. Furthermore, some smaller processes were presented that were modeled in BPMN due to a preceding recommendation. \newpage
At this point, we can answer the three research questions from the very beginning of this thesis. The first question deals with the DMN notation and its ability to model decisions and replacing gateway modeling in BPMN. In Chapter \ref{chapter:combination} we found a way to provide a replacement of BPMN gateways and even more elements, such as finding the right assignee for a task or listening to events. DMN serves as a complementary notation, supporting BPMN by separating the concern of data handling from larger processes while making them clearer and easier to understand. The process execution might also feature a performance increase, which could be part of future work. \\
The second question deals with CMMN and the usage of cases in a software project. During the investigation of the \textit{Company Audit} process, we applied the indicators on the super-process and some sub-processes leading to several CMMN recommendations. All of the CMMN processes originate from a software development project and will be implemented in a middleware. This middleware serves as an orchestration for the portal and connects the frontend with the database layer. The implemented process consequently orchestrates the interaction between database, frontend and other systems like access rights management.\\
CMMN is a suitable way to support a system when it comes to human intervention. The auditor, for instance, is a worker that interacts with the system in a semi-automated way, because he handles the necessary checks to prove a company's authenticity. Furthermore, there will be use cases where the user, instead of a machine, is in the center of action. These processes can conveniently be modeled in CMMN, connecting automated tasks with manual ones and create a unique user experience based on partly predefined processes.\\
The last research question deals with the combination of the notations and beneficial aspects such as improved readability and information flow. Concerning the latter, this thesis could not investigate the implementation of processes and corresponding improvements about performance and information flow. Since the eKultur project still is in the design phase, nothing has been implemented while this thesis was written. None of the processes have been executed once, neither has the middleware been set up during the investigation of the processes. All mentioned processes have been modeled initially during this thesis, resulting in a lack of comparable objects, particularly process models, and data for instance cycle times or information congestion. However, the combination of models itself was investigated in detail as well as the aspect of improved readability. Especially the latter aspect can be approved by Chapter \ref{chapter:combination}, as long as the \textit{Separation of concerns} paradigm is being followed by the process analysts.\newpage It is inherent to encapsulate a single logical unit into a sub-process and orchestrating the whole set of sub-processes with a super-process. This achieves improved readability and simplifies the implementation tasks for developers. Due to smaller logical units, developers are able to create classes featuring a high cohesion and loose coupling. These classes are also characterized by improved readability and encapsulating a single logical unit. Ideally, developers can derive the code to be implemented from a complete process model with all the attributes and references set during the design phase. The combination of process models can help to improve the interface design in a technical perspective, as well as aiding the frontend engineers to decide when a user is intended to do more manual work or has to reply on question the system presents him. 

\section{Downsides of this thesis}
The major part of this thesis deals with the creation of a methodology to identify a suitable modeling language for a discovered process. Chapter \ref{chapter:indicators} and \ref{chapter:combination} carried out this intention by reviewing literature and aggregating information about the modeling notations. Consequently, the research questions could be answered. \\
However, Chapter \ref{chapter:case_study} was intended to prove the indicators, which could only be achieved to a certain extent. As part of the eKultur project and author of this thesis, a dual role was created that has lead a comparison problem. As a consequence, there is no comparable material that can be used to validate an improvement by applying the indicators to the models and re-modeling them. For this thesis, models created by different analysts should have been used to increase the provability of the methodology. \\
Another aspect is the lack of statistical data. The scope of this thesis was not to do a statistical evaluation resulting in a set of data that could be used for a proof of concept. Neither was the scope to acquire processes from different companies using them for the practical application. However, both could be part of future work that deals with a comparison.
Eventually, the practical application part of Chapter \ref{chapter:case_study} resulted in a show case on how to apply the indicators and the combination on processes along with the usage of the decision engine for the recommendation. As the title states, this thesis functions as a \textit{Guideline to combine and differentiate between BPMN, CMMN and DMN}, which was realized in the preceding chapters. The application part showed that the indicators are working for process discovery products, for instance process maps and documentation. \newpage Since eKultur uses manual process discovery, automated process discovery could not be tested. This might also be part of future work. \\
Conclusively, we were able to show a working product for processes that were manually discovered, but could not prove the concept on a larger scale. 

\section{Future work}
As stated in the section above, the large scale proof could not be achieved. Neither could the indicators be applied on results of automated process discovery. Future work could deal with a statistical evaluation of the 
 application, as well as include results from automated process discovery. Another aspect is the comparison between models that have been created without the indicators and ones implementing the recommendation system. This could measure the robustness of the concept. \\
Furthermore, the guideline in general could be tested empirically. People without knowledge about process modeling could use the guideline to learn about the three different modeling notations and be advised to create a process model from a given process documentation. Afterwards, the effectiveness of the guideline could be evaluated. Business analysts could be asked to use the guideline and give their feedback about the concept and how they would characterize certain aspects of the notations and the processes they capture. This could improve the indicators and the recommendation system. Also, the weighting seen in Table \ref{tab:indicators} is a value that has been set according to the importance of certain aspects. Future work could check the weighting and adjust it according to statistical evaluation. \\