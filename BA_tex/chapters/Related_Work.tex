\chapter{Related Work}
\label{chapter:related work}

The combination of and differentiation between different modeling techniques has always been an issue in BPM. 
A common approach is evaluating the usage of specific notations. \cite{LuSadiq} investigates the usage of "Graph-Based Process Modeling Approaches", such as ActivityFlow, \ac{YAWL} or FlowMake. Moreover, they checked "Rule-Based Process Modeling Approaches" and environments, specifically AgentWork, ADEPT and Object-Rule Approach. To compare these methodologies, a criteria framework needed to be created. The framework covers five categories: Flexibility, Expressibility, Adaptability, Dynamism and Complexity \cite{LuSadiq}. Instead of a comparison of two or three modeling notations, their research is based on a high level view and deals only with a comparison of two major approaches. This lead to a strength and weakness analysis of each approach. \\
Another comparison and categorization of Business Process Management languages is provided by \cite{KoLeeLee2009}. The authors first categorize over 30 modeling languages into four main groups: "execution, interchange, graphical, and diagnosis standards" \cite{KoLeeLee2009}. With the introduction of 30 notations, the paper gives an overview of almost every important modeling languages at this point in time. They also identify gaps and misconceptions of each modeling language, future trends, and the necessity of standardization diagnosis languages such as \ac{BPRI} and \ac{BPQL}. \\
Not only modeling languages can be compared, but also their reference models. Reference models are more abstract ways to describe the characteristics of a modeling language and express the technical interaction of the language's elements. The Objective Management Group, for instance, uses UML to illustrate the technical interactions of elements, specifically by modeling the attributes and inheritances. \cite{FettkeLoosZwicker2006} investigates over 30 reference models using a framework based on criteria "[...] such as application domain, used process modeling languages, model’s size, known evaluations and applications of process reference models." The proposed framework can be used either to compare existing reference models, or to "guide the development of new ones" \cite{FettkeLoosZwicker2006}. \\
In general, frameworks based on criteria or categorization are often used to compare modeling languages. \cite{BendraouJezequelGervaisEtAl2010} defines a framework to compare six UML-based languages. Their framework is built on requirements such as "semantic richness, modularity, executability [...]" and more \cite{BendraouJezequelGervaisEtAl2010}. \cite{Vallecillo2010} has a similar approach by defining a framework for \textit{Domain Specific Modeling Languages}, used in "Model-based Engineering." \\
To create a framework, another framework is necessary. \cite{SoederstroemAnderssonJohannessonEtAl2002} provides a general concept to compare modeling languages on a more abstract level. The authors developed a guideline to create frameworks for comparison, based on simple questions like "what, how, why, who, when, and where" \cite{SoederstroemAnderssonJohannessonEtAl2002}. \\
Besides surveys and frameworks to differentiate modeling languages or reference models, the categorization is also a common methodology for comparisons. \cite{MiliTremblayJaoudeEtAl2010} classifies modeling languages such as BPMN, UML, \ac{WSBPEL} and more to define, how they had been standardized and formalized. They were also compared in terms of description, analysis and enactment. \\
\cite{RoserBauer2005} not only focuses on the categorization of modeling notations, but also applies it to an example which is very similar to this thesis. The authors intent was provide a framework for selecting the appropriate notation for a given process in a service-oriented environment. 