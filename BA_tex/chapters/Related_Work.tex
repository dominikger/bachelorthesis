\chapter{Related Work}
\label{chapter:related work}

\section{BPMN}
 -- todo 
 
 Geplant: Am Ende der Ausarbeitung 
 
 
\section{CMMN}
Case Management and knowledge work are not brand new inventions that have been created in the past few years. "Peter F. Drucker made the first reference to knowledge work in (...) 1959 (...)" \cite{Motahari-NezhadSwenson2013}. A current "overview and research challenges" provide \cite{Motahari-NezhadSwenson2013} who explain the difference between business process management and adaptive case management. They briefly sum up the state of the art in case management technology and the next generation solutions. 
Mentioning technology and tools for Case Management, CMMN and Adaptive Case Management, there are many articles dealing with these topics. \cite{OsuszekStanek2015} describe how adaptive case management can be implemented in businesses and integrated in Enterprise Resource Planing systems (ERP). Additionally they approach a new architecture which decouples decision logic, knowledge work and process flows. All this leads to a better handling of information and an optimization of business modeling. 
Another practical example provide \cite{Kuzin2013} explaining the company's approach towards an implementation of the CMMN paradigm. This includes the ability to change requirements or orders during run-time, which is one of the major aspects in their system. To achieve this goal, they first set up a meta model of their order-based system and enhanced it afterwards. 
These practical examples are important in order to evaluate the compatibility with the CMMN specification and other modeling languages, specifically BPMN. They also provide a good overview of how to combine modeling techniques and how they are realized as a system in companies. 
A more theoretical approach to case management and CMMN particularly provide \cite{WangTraore2014} and \cite{Zeising_2014}. They both do research on transforming CMMN into different languages. \cite{WangTraore2014} do model-to-model transformation from CMMN to  DDML (DEVS-driven Modeling Language) which is used to formalize CMMN and analyze it afterwards. \cite{Zeising_2014} have a similar approach, but a different goal. Due to weaknesses of CMMN, the language cannot be used to create a platform for both agile and route processes. They describe agile processes as the ones "(...) of which the exact flow cannot be determined completely a priori" \cite{Zeising_2014}, which is a fundamental characteristic of knowledge work and the reason why case management is so important for many industries. Coping with CMMN's downsides they build their platform on a "rule-based cross-perspective and model intermediate language on textual basis, (...) called \textit{Declarative Process Intermediate Language (DPIL)}" \cite{Zeising_2014}. \\
A useful source for evaluating CMMN as a standardization for adaptive case management is \cite{KurzSchmidtFleischmannEtAl2015}. The subtitle \textit{Examining the applicability of a new OMG standard for adaptive case management} is a good foundation to see how OMG met the expectations from the industry and researchers. This paper sets up requirements deriving from different sources described in detail in section two \cite{KurzSchmidtFleischmannEtAl2015}. At the end of their paper, they evaluate how good the requirements were fulfilled by the CMMN standardization and provide feedback for future improvements. 

\section{DMN}
The Decision Model and Notation standardization was meant to improve the \textit{separation of concerns} \cite{BiardMauffBigandEtAl2015} which is the decoupling of decision logic and the control-flow. Biard et al. investigate how the new standard DMN can be used for decoupling BPMN and the decisions modeled as gateways. Decision-modeling is not typically included in control-flow oriented modeling languages. BPMN has not the power to model vast decision-trees due to the gateway restrictions. \cite{BatoulisMeyerBazhenovaEtAl2015} even calls it a "(...) [misuse] for modeling decision logic". They found an autonomous way of separating the concerns. After averaging more than 900 models from different industries they introduced a "(...) semi-automatic approach to identify decision logic in business processes (...)" \cite{BatoulisMeyerBazhenovaEtAl2015}. This semi-automatic approach incorporates the 900+ models they used to identify patterns in decision modeling. Formalization is not one of the key issues in this thesis, but the translation of BPMN to DMN or the link between them definitely is. 
Evaluating the compatibility of DMN with different modeling languages has been the objective of \cite{MertensGaillyPoels2015}. They approached a combined solution for knowledge-intensive work modeling and extracting the decision logic, what lead them to a new language called \textit{Declare-R-DMN}. Although the Declare language is not part of this thesis, the combination of it with DMN is useful to evaluate the compatibility with BPMN and CMMN. 