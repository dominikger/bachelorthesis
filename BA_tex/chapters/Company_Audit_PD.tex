
\chapter{Company Audit Process Documentation}
\label{app:A}
\begin{otherlanguage}{german}

\section*{Prolog}
Der Prozess prüfen ist entweder ein
\begin{itemize}
\item Prüfprozess, der innerhalb der Community mit der Applikation unterstützt wird oder
\item Eine manuelle Prüfung
\end{itemize}

\section*{Ziel des „Betrieb-prüfen“ Prozesses}
Ziel des „Betrieb prüfen“ Prozesses ist es, die Qualität der Betriebe, die im eKulturPortal miteinander agieren, zu identifizieren (Echtheit). 

\section*{Herangehensweise}
Folgende Qualitätsmerkmale könnten geprüft werden, wobei nicht alle Kriterien übereinstimmen müssen: Beispielsweise wird eie Firmeneigenschaft ggf. in keinem amtlichen Verzeichnis geführt. Der zu prüfende Betrieb ist somit eine Körper\-schaft oder eine öffentliche Einrichtung \footnote{Das eKulturPortal betrachtet zum Beispiel ein Kulturamt oder die Baubehörde einer kommunalen Verwaltung als eigenen Betrieb. Hier ist darauf zu achten, dass die Verwaltung selbst in den Verzeichnissen gesucht wird. Beispiel: Forum Unterschleißheim wird über Rathaus Unterschleißheim gefunden.}.

Folgende Informationen sind Pflichtfelder und sind prüfbar:
	\begin{itemize}
	\item Firmensitz
	\item Postanschrift
	\item Geschäftsführer oder EigentümerIn
	\end{itemize}

\begin{itemize}
\item Sie ist in einem amtlichen Verzeichnis geführt
	\begin{itemize}
	\item Bundesanzeiger für Justiz: Nachweis durch die ordentliche Buchführung
	\item Amtsgericht/Registergericht (Handelsregister): Handelsregisternummer (Online Prüfung):
		\begin{itemize}
		\item Auswahl: Normale Suche,
		\item Eingabe Firmenname
		\item Firma wurde nicht gefunden = „Ihre Suche hat 0 Treffer ergeben = Prüfung negativ
		\item Firma wurde gefunden: Ergebniserwartung „Status=aktuell“
		\end{itemize}
\item Eine DUNS-Nummer ist vergeben, wenn der Betrieb in einem amtlichen Verzeichnis geführt wird
	\begin{itemize}
	\item Eingabe Firmenbezeichnung, Land des Betriebssitzes
	\item Firma wird nicht gefunden = Prüfung negativ
	\item Firma wird gefunden = Prüfung positiv
	\item Abgleich der Informationen 
		\begin{itemize}
		\item Für öffentliche Einrichtungen: kein Qualitätsmerkmal, da DUNS zwar die Aktualität zusagt, jedoch bei einer Stichprobe mit drei Kommunen die Angaben zum Bürgermeister veraltet waren.
		\item Für Körperschaften muss übereinstimmen, da die UPIK Informationen aus dem Handelsregister erwirbt
		\end{itemize}		
		\end{itemize}

	\end{itemize}
\item Der zu prüfende Betrieb ist nicht über die Eigentümerschaft spezifiziert (zum Beispiel muss keine Körperschaft sein)
	\begin{itemize}
	\item Gläubiger ID 
		\begin{itemize}
		\item Ist vorhanden = Prüfung positiv
		\item Gläubiger ID hat 18 Stellen
		\item Technische Prüfung\footnote{Prüfung ist nur möglich, in dem eine Überweisung an die zu prüfende Firma durchgeführt wird. Hier übertragen die Banken über den Überweisungsnachweis automatisch die Gläubiger ID (analog Verfahren PayPal: 0,01€ Überweisung)} = korrekt  positives Merkmal\\
		
		\end{itemize}
	\item Website
	\begin{itemize}
		\item Website ist vorhanden
		\item Impressum ist abgebildet mit der Angabe der Eigentümerschaft = Eingaben stimmen überein  positiver Status
	\end{itemize}
	\item Websuche: Ergebnisse zum Betrieb werden gefunden 
		\begin{itemize}
		\item Aktualität = aktuelle und zukünftige Ereignisse werden gefunden = positives Merkmal
		\end{itemize}
	\item Menschliches Ermessen
		\begin{itemize}
		\item Betrieb ist der eKultur GmbH persönlich bekannt
		\item Schriftliche Begründung durch Prüfer
		\item Betrieb wird von bereits registrierten und selbstverwaltenden Betrieben (Minimum 3) bestätigt. Hier erhält der Prüfer die Möglich\-keit, „Prüfen durch Community“ auszuwählen. Innerhalb einer definierten Frist erhält er hier über die Community die Echtheit be\-he her stätigt. Diese Auswahl ist vor allem bei jenen Betrieben ein Muss-Prozess, die nicht als Körperschaft agieren. Bei Körperschaften liegt diese Prüfung im Prüferermessen.
		\item Voraussetzung (vorgelagerter Prozess): Bei Erfassungsprozess durch den Betriebserfasser muss hierfür bei nicht Körperschaften 5 Referenzen aus dem eKulturPortal ausgewählt werden.
		\end{itemize}
	\end{itemize}
\end{itemize}

Zu klären: sollen Prüfungen auch auf Angebote für Räumlichkeiten (Probenraum, Veranstaltungsraum) erstellt werden? Hier ist die Überlegung, einen Qualitätsbalken einzublenden: Veranstaltungen sind hier schon durchgeführt worden, Proberäume wurden bereits angemietet und der Mietzeitraum liegt in der Vergangenheit bzw. die Rechnung wurde bezahlt.
\end{otherlanguage}