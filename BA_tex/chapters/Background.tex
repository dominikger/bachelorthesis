\chapter{The Evolution from BPMN to CMMN and DMN }
\label{chapter:Background}

\section{BPMN and its shortcomings}
\subsection{Origins of BPMN}
The evolution of business processes and process thinking dates back to the 1980s when Michael Porter developed the \textit{Value Chain} making a first proposal on how to align companies along the business processes \cite{Porter1988}. 
This major step initiated consecutive research and improvements in the field of business process engineering and re-engineering. The latter has been a very popular technique for companies to strip down their legacy processes, optimize them and implement the new ones. A very prominent example is Ford who copied the Mazda's concept of a central database replacing an old fashioned paper stream \cite{Dumas2013}. \\
Inspired by Ford and Mazda, the \textit{Business Process Redesign} (BPR) was emerging in until the late 1990s. At that time, companies tried to redesign in a radical manner the way processes, people and data keeping works. Dumas et al. define four reasons for the fall of BPR, whereof over-radicalism might be the most influential one \cite{Dumas2013}. \\

Despite the fall of BPR, the management of business processes became a prominent part of business optimization. Consequential different methodologies like the \textit{BPM lifecycle} or the \textit{Architecture of Integrated Information Systems} (ARIS) were born along with new modeling languages. "ARIS [also] provides a modeling language known as event-driven process chains (EPCs)" \cite{Lankhorst2009}. By today, ARIS still is a very popular, thus a well known, methodology for enterprise modeling and EPCs a widely used modeling language for event-driven processes, as the name states. \\
ARIS was published 1994 by August Wilhelm Scheer, a researcher for Business Information Systems at the Saarland University. Around ten years later, Stephen A. White and the \textit{Business Process Management Initiative}, consisting of roughly 35 individuals and companies total, published the first version of the new \textit{Business Process Modeling and Notation} (BPMN) concept \cite{Allweyer2010}. In 2006, the OMG adopted BPMN as well as the BPMI and published an overhauled version, BPMN v. 2.0 in 2011. This version was highly anticipated and had to meet high expectations. The goal was to create a language with the ability to interchange modeling tools. New parts were added such as the \textit{Choreography diagram} for modeling data exchange between partners and the \textit{Conversation diagram} showing the relationship between several partners \cite{Allweyer2010}. 

\subsection{Shortcomings} 
"As the name already indicates, BPMN is restricted to process modeling [...]" \cite{Lankhorst2009}. The goal in terms of BPMN was to create a powerful tool for modeling traditional processes. BPMN is an interchangeable language predestined as a middleware between the database and software layers. It has the ability to be executed \textit{Business Process Execution Language} (BPEL) and a well-known set of symbols and artifacts for modeling the diagrams. Additionally, it inherited its simplicity from EPCs and the \textit{Unified Modeling Language} (UML), which is also part of the OMG family. \\
This simplicity, however, has limitations when it comes to agile and decision intensive models. Decisions, for instance, are illustrated by gateways. These gateways work as connectors separating the process flow into different branches. Each branch has a token wandering in flow direction through the graph. But what if a business wants to change decisions in real time without losing the tokens flowing through the chart? At the current state of development, which is BPMN v 2.0.2, there is no possibility to do so. This is only one of the problems which \cite{Repa2014} and \cite{ReckerIndulskaRosemannEtAl2006} investigated more closely. They both came to the conclusion that BPMN is lacking the ability implement business rules which leads the user to display them in spreadsheets. 
Another downside is the impossibility of modeling agile processes. People working in departments which do not have strict processes but more agile ones cannot use the supportive power of BPMN or process optimization. Despite their knowledge intensive work, they still can benefit from optimizing workflows, or even help new employees getting known to their new job in the department. In this thesis, we will not focus on BPMN's syntax deficits but more on the possibility CMMN and DMN offer. 

\section{CMMN}