% German abstract for the CAMP report document
% Included by MAIN.TEX


\clearemptydoublepage

\vspace*{2cm}
\begin{center}
{\Large \bf Zusammenfassung}
\end{center}
\vspace{1cm}

In den Jahren 2014 und 2015 veröffentlichte die Object Mangagement Group (OMG) zwei Standardisierungen neuartiger Modellierungssprachen, \textit{Case Management Model and Notation} (CMMN) und \textit{Decision Model and Notation} (DMN), welche von der Business Process Management Branche schnell aufgenommen wurden. Beide Modellierungssprachen bieten erstmalig die Möglichkeit, flexiblere und sogar agile Arbeitsprozesse abzubilden, anstatt sich nur auf routinierte und vorhersehbare zu beschränken. \\
Da die Standardisierungen relativ neu auf dem Markt sind, mangelt es bislang an einer detaillierten Richtlinie zur Kombinierung und Differenzierung zwischen den genannten Modellierungssprachen und der sehr bekannten \textit{Business Process Model and Notation} (BPMN) Methodik. 
Sowohl die Recherche und Zusammenstellung aktueller Forschungsliteratur zu BPMN, CMMN und DMN, als auch zu Methoden der Prozessidentifikation auf manuelle sowie automatisierte Weise sind genauso Kern dieser Thesis, wie die Untersuchung der publizierten drei Standards BPMN, CMMN und DMN. Darüber hinaus wird jede dieser Modellierungssprachen anhand von Beispielen und bekannten Mustern den Lesern nähergebracht. \\
Diese Bachelorarbeit befasst sich mit dem Entwurf einer Methodik basierend auf Indikatoren, die Ergebnisse aus der Prozessidentifikationsphase aufgreift und eine geeignete Modellierungssprache für die Modellierung empfiehlt. Weiterhin werden bewährte Muster für Prozessidentifikation, die Kombinationsmöglichkeiten der Modellierungssprachen in technischer, als auch graphischer Hinsicht untersucht und anschließend auf Prozesse eines Softwareprojekts in der Theaterindustrie angewendet. \\
Diese Fallanwendung zeigt, dass die Indikatoren ein probates Mittel zur Unterscheidung zwischen BPMN, CMMN und DMN sind. Sie zeigt auch, dass sich CMMN nicht nur zur reinen Fallmodellierung eignet, sondern auch in der Softwareentwicklung durchaus Anwendungen finden kann. Letztlich zeigt die Falluntersuchung auch, dass DMN wenig Spielraum für eigenständige Modelle bietet und vielmehr einen komplementären Charakter besitzt. 
