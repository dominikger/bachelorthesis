% Abstract for the TUM report document
% Included by MAIN.TEX

\clearemptydoublepage
\phantomsection
\addcontentsline{toc}{chapter}{Abstract}	

\vspace*{2cm}
\begin{center}
{\Large \bf Abstract}
\end{center}

%\textbf{Introduction} 
In the past two years, the Object Management Group (OMG) released two, by the industry highly anticipated modeling specifications: \textit{Case Management Model and Notation} (CMMN) in 2014 and the auxiliary notation \textit{Decison Model and Notation} (DMN) in 2015. Both specifications allow analysts to create more flexible and even agile processes instead of only routine and predictable ones. Especially the newly added run-time adjustment and planning is in the center of research along with its realization. \\
%\textbf{Purpose}
Since the specifications are relatively new to the market, a recommendation methodology based on indicators to differentiate between the notations, as well as a guideline to combine CMMN, DMN and BPMN has been missing. \\
%\textbf{Method}
For this reason, indicators were constructed to distinguish between each language and recommend a suitable one for a given process discovery output. The indicators derived from an examination of the OMG's specifications, additional literature and examples taken from the theater production industry. Furthermore, several techniques and strategies as well as best practices for manual and automated process discovery are provided. Practical examples help the reader to get familiar with each modeling specification.\\ 
%Derived from an examination of the specification, additional literature and practical examples, indicators were constructed to distinguish between each language and eventually recommend a suitable one for a given process discovery output. Furthermore, several techniques and strategies as well as best practices for both, manual and automated process discovery are provided. Practical examples help the reader to get familiar with each modeling specification.\\
%\textbf{Product}
As a result, processes taken from the eKultur GmbH provide a guideline on how to use the recommendation engine and applying the methodology in a software engineering environment. Additionally, the usage of process discovery strategies is explained.
%\textbf{Conclusion}
\\To conclude, the case study shows that these indicators serve as robust way to distinguish between BPMN and CMMN. Moreover it demonstrates CMMN's applicability in software development projects along with BPMN, which has been and still is a major modeling language when it comes to implementing business logic. Additionally it showed the auxiliary character of DMN as a modeling notation that works well with BPMN, but has very few possible applications as a standalone one.\\

